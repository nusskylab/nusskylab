%% ----------------------------------------------------------------
%% Thesis.tex -- MAIN FILE (the one that you compile with LaTeX)
%% ---------------------------------------------------------------- 

% Set up the document
\documentclass[a4paper, 11pt, oneside]{Thesis}  % Use the "Thesis" style, based on the ECS Thesis style by Steve Gunn
% \graphicspath{./Images/}  % Location of the graphics files (set up for graphics to be in PDF format)

% Include any extra LaTeX packages required
\usepackage[square, numbers, comma, sort&compress]{natbib}  % Use the "Natbib" style for the references in the Bibliography
\usepackage{verbatim}  % Needed for the "comment" environment to make LaTeX comments
\usepackage{vector}  % Allows "\bvec{}" and "\buvec{}" for "blackboard" style bold vectors in maths
\hypersetup{urlcolor=blue, colorlinks=true}  % Colours hyperlinks in blue, but this can be distracting if there are many links.

%% ----------------------------------------------------------------
\begin{document}
\frontmatter      % Begin Roman style (i, ii, iii, iv...) page numbering

% Set up the Title Page
\title  {Skylab: NUS Orbital Project Platform}
\authors  {\texorpdfstring
	{\href{mailto:franklingujunchao@gmail.com}{Gu Junchao}}
	{Gu Junchao}}
\addresses  {\groupname\\\deptname\\\univname}  % Do not change this here, instead these must be set in the "Thesis.cls" file, please look through it instead
\date       {\today}

\maketitle
%% ----------------------------------------------------------------

\setstretch{1.3}  % It is better to have smaller font and larger line spacing than the other way round

% Define the page headers using the FancyHdr package and set up for one-sided printing
\fancyhead{}  % Clears all page headers and footers
\rhead{\thepage}  % Sets the right side header to show the page number
\lhead{}  % Clears the left side page header

\pagestyle{fancy}  % Finally, use the "fancy" page style to implement the FancyHdr headers

%% ----------------------------------------------------------------
% Declaration Page required for the Thesis, your institution may give you a different text to place here
\Declaration{

\addtocontents{toc}{\vspace{1em}}  % Add a gap in the Contents, for aesthetics

I, Gu Junchao, declare that this thesis titled, `Skylab: NUS Orbital Project Platform' and the work presented in it are my own. I confirm that:

\begin{itemize} 
\item[\tiny{$\blacksquare$}] This work was done wholly or mainly while in candidature for a research degree at this University.
 
\item[\tiny{$\blacksquare$}] Where any part of this thesis has previously been submitted for a degree or any other qualification at this University or any other institution, this has been clearly stated.
 
\item[\tiny{$\blacksquare$}] Where I have consulted the published work of others, this is always clearly attributed.
 
\item[\tiny{$\blacksquare$}] Where I have quoted from the work of others, the source is always given. With the exception of such quotations, this thesis is entirely my own work.
 
\item[\tiny{$\blacksquare$}] I have acknowledged all main sources of help.
 
\item[\tiny{$\blacksquare$}] Where the thesis is based on work done by myself jointly with others, I have made clear exactly what was done by others and what I have contributed myself.
\\
\end{itemize}
 
 
Signed:\\
\rule[1em]{25em}{0.5pt}  % This prints a line for the signature
 
Date:\\
\rule[1em]{25em}{0.5pt}  % This prints a line to write the date
}
\clearpage  % Declaration ended, now start a new page

%% ----------------------------------------------------------------
% The "Funny Quote Page"
\pagestyle{empty}  % No headers or footers for the following pages

\null\vfill
% Now comes the "Funny Quote", written in italics
\textit{``If debugging is the process of removing software bugs, then programming must be the process of putting them in.''}

\begin{flushright}
Edsger Dijkstra
\end{flushright}

\vfill\vfill\vfill\vfill\vfill\vfill\null
\clearpage  % Funny Quote page ended, start a new page
%% ----------------------------------------------------------------

% The Abstract Page
\addtotoc{Abstract}  % Add the "Abstract" page entry to the Contents
\abstract{
\addtocontents{toc}{\vspace{1em}}  % Add a gap in the Contents, for aesthetics

Orbital is the School of Computing`s self-driven programming summer experience. It is designed to give first-year students the opportunity to self-learn and build something useful\cite{citation0}. As more and more students join Orbital program, administration of the program and evaluations among teams are becoming increasingly challenging. Therefore, Skylab is designed and implemented to help students to submit submissions and evaluate other teams with ease as well as to help the administrator of Orbital program overlook and manage this module well. In this project, we find and solve real-life problems faced by students, advisors and administrators of Orbital program with an extensible system design and continuous contribution in an agile software development process.

\begin{flushleft}
{\normalsize \textbf{Subject descriptors:} \par}
{\normalsize \subjectname \par}
{\normalsize \textbf{Keywords:} \par}
{\normalsize \keywordnames \par}
\end{flushleft}
}

\clearpage  % Abstract ended, start a new page
%% ----------------------------------------------------------------

\setstretch{1.3}  % Reset the line-spacing to 1.3 for body text (if it has changed)

% The Acknowledgements page, for thanking everyone
\acknowledgements{
\addtocontents{toc}{\vspace{1em}}  % Add a gap in the Contents, for aesthetics

I would like to express my most sincere appreciation to my project supervisor and project
administrator, Dr. Min-Yen Kan. Throughout this project, it was him who tirelessly
provided me with significant support and assistance. I would also like to appreciate Orbital Program advisers and students for suggesting many useful features and bringing up issues to make Skylab more usable.

}
\clearpage  % End of the Acknowledgements
%% ----------------------------------------------------------------

\pagestyle{fancy}  %The page style headers have been "empty" all this time, now use the "fancy" headers as defined before to bring them back


%% ----------------------------------------------------------------
\lhead{\emph{Contents}}  % Set the left side page header to "Contents"
\tableofcontents  % Write out the Table of Contents

%% 
% End of the pre-able, contents and lists of things
% Begin the Dedication page

\setstretch{1.3}  % Return the line spacing back to 1.3

\addtocontents{toc}{\vspace{2em}}  % Add a gap in the Contents, for aesthetics


%% ----------------------------------------------------------------
\mainmatter	  % Begin normal, numeric (1,2,3...) page numbering
\pagestyle{fancy}  % Return the page headers back to the "fancy" style

% Include the chapters of the thesis, as separate files
% Just uncomment the lines as you write the chapters

\chapter{Introduction}

Orbital is the School of Computing’s self-driven programming summer experience. It is designed to give first-year students the opportunity to self-learn and build something useful. It is designed as a 4 modular credit (MC) module that is taken pass/fail (CS/CU) over the summer. With its focus on hands-on experience, it has been catching more and more attention and an increasing number of year-one students are joining the program to code something useful and interesting. During the academic year of 2015-2016, more than 250 students completed Orbital program.

For the evaluation of Orbital program, students are supposed to submit to Milestones as a team, stating what they have done during that phase. And then assigned peer teams(each team will be assigned for about 3 peer teams) will be giving feedback regarding the submission and the application built by the team of students. At the same time, there will be an adviser who will overlook the whole process and provide evaluation of a team's submission too. After 3 submissions and evaluations are done, a feedback is expected from a team to its peer teams and its adviser regarding the quality of evaluations received.

The nature of Orbital defines the scope of Skylab —- A development project built for peer evaluations. It also provides students with a real-life Software Engineering training ground to learn and sharpen programming and system design skills. 

\section{Challenges}

\subsection{System design}
Skylab is built on top of Ruby on Rails, a mature convention-over-configuration web framework. So the first challenge for me is to get familiar with the conventions and recommended ways of doing things in Rails community. Then I can design the web application on the top of main-stream philosophy in Rails community. Another issue with the design of Skylab is that this is the first time I have been designing such an application from ground up, without any guidance from any experienced Ruby on Rails developers. Therefore, it is all about try-and-error and explore my own way. Reading books and browsing on-line tutorials helped me a lot and luckily there are plenty of resources about Ruby on Rails development due to its popularity.
\subsection{Evolving requirements}
Although the scope of the project is very clear and well-defined, changes in requirements are expected and did happen a lot due to lack of consideration during initial system design phase, evolving features of Skylab project and Skylab users' feedback. The challenge is to cope with all changes and make necessary adjustments to the Skylab as new requirements are brought up. Therefore agility in development and adaptability of the system is expected.
\subsection{Security}
% Security is definitely a very important perspective in web development. Although Rails is handling security well by,
To be continued
\subsection{Data migration}
Sometimes schema migration is required as a result of change in requirements. Therefore, dealing with old data and migration of data without affecting the use of application is another challenge during the development of Skylab. Extreme attention when migrating is required as data may not be clean enough and careless migration may even cause the system to be unusable for some users.
\subsection{Coding quality/maintainability}
Although currently there is only me constantly contributing to Skylab repository, a good development cycle which is agile enough is not only important for me to keep track of history and manage different issues but also convenient for developers who will join later to jump in and get started. Testing is also a very important factor when it is comes to long-term maintenance. A continuous integration would also help me in catching regression errors in development early and easily. Besides all mentioned above, refactoring is helping in the growth of the project.


\section{Objectives}

To be continued

\section{Outline}

In this report, I will discuss various accomplishments I have done in the development. Chapter 2 will be an overview of current architecture of Skylab, and methodologies I employed during the development. Then security related issues such as user authentication and access control will be discussed in Chapter 3. Chapter 4 is about problems in the implementation of submissions. Chapter 5 will be talking about design for peer evaluations. After then feedback related details will be discussed in Chapter 6. Chapter 7 is about adviser focus group meeting and its findings. Last but not least, a summary of work and an overlook of future development will come in Chapter 8.


 % Introduction

\chapter{Background}

Skylab is built on top of Ruby on Rails, a well-known web development framework, with a great support from a large community, used widely in the industries by companies like Twitter, Groupon, Bloomberg, Airbnb and many more\cite{citation1}. There are many reasons why we have chosen Ruby on Rails. Firstly, Ruby is clean elegant and easy to read and this feature enables programmers to be more productive. Secondly, Ruby on Rails is adopting many advanced industrial conventions and this enables contributors to have good exposure to programming in the industry. What is more, scaffolding and many gems can significantly boost the productivity. Last but not least, Ruby on Rails community has a favor for open source contribution which aligns well with the open source nature of Skylab.

For the selection of database, we used PostgreSQL. Part of the reason is that it is open source and quite mature, with good drivers available in many languages\cite{citation2}. Besides, we need full ACID compliance for consistency of data and we do not need scalability in the foreseeable future. And PostgreSQL has recently added implementation for rich data structures such as JSON which would make development much easier\cite{citation3}.

Puma is the web server we have chosen for Skylab. Among Passenger, Unicorn, Rainbows! and Puma, Puma is considered to be fast and memory friendly according to on-line benchmarking\cite{citation4}. Puma is built for high-concurrency and speed and more and more developers to switching in Rails community\cite{citation5}.

Nginx has grown in popularity since its release due to its light-weight resource utilization and its ability to scale easily with low memory usage. It excels at serving static content quickly and is designed to pass dynamic requests off to other software that is better suited for those purposes\cite{citation6}. There are also some benchmarking results that indicate the superiority in Nginx handling concurrent access and low memory usage of Nginx\cite{citation7}. Therefore, Nginx has been chosen as our HTTP server.

\section{System design}

The basic MVC structure of Rails is shown in the figure below:
\begin{figure}[h]
\caption{Illustration of how MCV works in Rails}
\centering
\includegraphics[width=0.7\textwidth]{MVC_Rails}
\end{figure}

\section{Development process}

To be continued % Background

\chapter{Submission}
 
\section{Handling of rich text}

\section{Usability}
 % Submission & Peer evaluation & Feedback and question system

\chapter{Peer Evaluation}
 
 After students have submitted to milestones, peer evaluation process can begin. Teams will look through evaluated teams' projects and submissions and evaluate their performance in \textit{Peer Evaluation}, which is a very important component in determining whether the evaluated teams can pass or not. Although there are different questions for each peer evaluation, all evaluations contains essentially 2 parts:

 \begin{itemize}
  \item Public: a section with general feedback on how well the evaluated team has done and the response will be viewed by target team with evaluator team name available.
  \item Private: a section with critiques and overall rating and critiques will only be viewed by target team without any evaluator team information while overall rating is only for grading purpose and not viewable by target teams.
\end{itemize}

\section{Loading of Peer Evaluation templates}

Currently, the public part of a \textit{Peer Evaluation} templates will be one html form and the private part is another one. This means all questions in the public part are coded in one predefined html template and same goes for questions in private part.

This approach does not seem to have much problem at first and it really served Skylab's purpose well by delivering perfectly workable features in time. However, since questions are different for different \textit{Peer Evaluation} templates, separate templates have to be created for new \textit{Peer Evaluation} template. And when user wants to view/submit/edit a \textit{Peer Evaluation}, the corresponding templates will have to loaded based on properties of the evaluation. In this way, the system is not really open to extension and clearly it violates Open-Close principle. After realizing this, a system for dynamically creating questions has been implemented for \textit{Feedback} and migration of \textit{Peer Evaluations} will be done as future work, which will be described in more details in Chapter~\ref{conclusionandfuturework}.
 % Security

\chapter{Feedback}

\textit{Feedback} is for evaluated teams to evaluate the \textit{Peer Evaluations} they have received and it is also a very important component in determining whether the evaluator team can pass or not. After realizing the lack of extensibility in the design for \textit{PeerEvaluations} and with more time available for implementation of \textit{Feedback}, a Survey Template and question creation system has been set up, which made the system open for extensions for more questions and even question types.

\section{Survey Template and question creation system}


 % Focus group meeting

\chapter{Conclusion and future work} \label{conclusionandfuturework}

\section{Conclusion}

Most core functionalities have been implemented for Skylab including Submissions, Peer Evaluations, Feedback. However, as pointed in previous discussions as well, issues in security, usability and system design do exist as well and those issues will be addressed in the the coming semester. Besides, more features are expected of Skylab as well to make it serve Orbital program better such as registration of interest, mailer as reminders for deadlines and logging of user activities.

\section{Future work}

A proposed set of major features to be completed in the future for Skylab:

\begin{itemize}
  \item Questions/template system(involving migration of current data): currently \textit{Feedback} is utilizing the \textit{SurveyTemplate} and \textit{Question} system but \textit{Peer Evaluation} and \textit{Submission} are still not. With migration to \textit{Questions} system we can further improve the system by adding more extensibility.
  \item Public view of projects \& teams \& students: listing of previous projects as evidence of work from alumni of Orbital program can also help to attract more freshmen joining.
  \item Registration and post Orbital feedback: As part of Orbital program, registration of interest and post Orbital feedback should be captured in Skylab as well.
  \item Implementation of cohort: Orbital is held every year and certainly Skylab is expected to take cohort into consideration and allow historical records to be captured in the database.
  \item Implementation of more mailer actions: we can explore power of emails by sending reminder emails to students when deadlines are approaching or even embed secure link for students to do submissions without login.
  \item Logging of user activities: by logging down activities carried out by different users, users can more easily figure out what has happened and get a better sense of the context of Skylab.
\end{itemize}

 % Conclusion

%% ----------------------------------------------------------------
% Now begin the Appendices, including them as separate files

\addtocontents{toc}{\vspace{2em}} % Add a gap in the Contents, for aesthetics

\appendix % Cue to tell LaTeX that the following 'chapters' are Appendices

% \input{Appendices/AppendixA}	% Appendix Title

%\input{Appendices/AppendixB} % Appendix Title

%\input{Appendices/AppendixC} % Appendix Title

\addtocontents{toc}{\vspace{2em}}  % Add a gap in the Contents, for aesthetics
\backmatter

%% ----------------------------------------------------------------
\label{Bibliography}
\lhead{\emph{Bibliography}}  % Change the left side page header to "Bibliography"
\bibliographystyle{unsrtnat}  % Use the "unsrtnat" BibTeX style for formatting the Bibliography
\bibliography{Bibliography}  % The references (bibliography) information are stored in the file named "Bibliography.bib"

\end{document}  % The End
%% ----------------------------------------------------------------
